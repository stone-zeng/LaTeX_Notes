%TODO 等的使用
%%TODO 内容问题
%%HACK 粗鄙技巧
%%CODE 代码改进

%TODO——注意事项
%CODE:20160629 每行 80 字符限制,中英文、数字之间加空格

\documentclass[oneside]{book}

%TODO——宏包
%%页面尺寸
\usepackage{geometry}
	\geometry{
		a4paper,
		left = 2.54 cm, right = 2.54 cm, top = 3.18 cm, bottom = 3.18 cm,
		headheight = 3 cm
	}

%%设置标题
\usepackage{titlesec}

%%交叉引用,超链接等
\usepackage[hyperindex]{hyperref}
	\hypersetup{
		%  PDF 书签
		bookmarksopen = true,
		bookmarksopenlevel = 1,
		bookmarksnumbered = true,
		%  PDF 标题作者
		pdftitle = {LaTeX 排版笔记},
		pdfauthor = {曾祥东},
		%脚注
%		hyperfootnotes = false,
		%目录  只引用页码
		linktoc = page,
		%超链接颜色
		colorlinks,
		linkcolor = {red!60!black},
		citecolor = {green!50!black},
		urlcolor = {blue!70!black}
	}

%%常规字体选择
%
%正文字体:Palatino
%无衬线字体A:Helvetica
%无衬线字体B:Source Sans Pro
%等宽字体:Courier
%
%以下两种方式已弃用:
%Helvetica: \usefont{T1}{phv}{m}{n}
%Courier: \usefont{T1}{pcr}{m}{n}
%
\usepackage[no-math]{fontspec}
	\setmainfont[
		Extension = .otf,
		BoldFont = texgyrepagella-bold,
		ItalicFont = texgyrepagella-italic.otf,
		SlantedFont = texgyrepagella-italic.otf
	]{texgyrepagella-regular}
	\setsansfont[
		Extension = .otf,
		BoldFont = texgyreheros-bold,
		ItalicFont = texgyreheros-italic.otf,
		SlantedFont = texgyreheros-italic
	]{texgyreheros-regular}
	\setmonofont[
		Extension = .otf,
		BoldFont = texgyrecursor-bold,
		ItalicFont = texgyrecursor-italic.otf,
		SlantedFont = texgyrecursor-italic
	]{texgyrecursor-regular}
	\newfontfamily{\SourceSans}{Source Sans Pro}
	\newfontfamily{\Songti}{方正书宋_GBK}

%%AMS 数学支持
\usepackage{amsmath}

%%Pi 符号
\usepackage{pifont}

%%调用 Unicode OpenType 数学字体
\usepackage{unicode-math}
	\setmathfont[
		math-style = ISO,
		bold-style = ISO
	]{texgyrepagella-math.otf}
%加粗使用 \symbf{}
%直立希腊字母:\uppi 等

%%中文文字处理
\usepackage[UTF8, heading = true]{ctex}
	\pagestyle{empty}
\usepackage{xeCJK}
	\setCJKmainfont[
		BoldFont = 方正书宋_GBK,
		ItalicFont = 方正楷体_GBK,
		Mapping = fullwidth-stop
	]{方正书宋_GBK}
	\setCJKsansfont[
		BoldFont = 方正黑体_GBK,
		ItalicFont = 方正黑体_GBK,
		Mapping = fullwidth-stop
	]{方正黑体_GBK}
%	\setCJKsansfont[
%		BoldFont = 思源黑体 CN Regular,
%		ItalicFont = 思源黑体 CN Regular,
%		Mapping = fullwidth-stop
%	]{思源黑体 CN Regular}
	\setCJKmonofont[
		BoldFont = 方正仿宋_GBK,
		ItalicFont = 方正楷体_GBK,
		Mapping = fullwidth-stop
	]{方正仿宋_GBK}
	\setCJKfamilyfont{宋体}{方正书宋_GBK}
	\setCJKfamilyfont{楷体}{方正楷体_GBK}
	\setCJKfamilyfont{黑体}{方正黑体_GBK}
	\setCJKfamilyfont{仿宋}{方正仿宋_GBK}

%%脚注增强版
\usepackage[stable, perpage, bottom]{footmisc}
	%需要调用 pifont 宏包
	%衬线加圈阳文数字:\ding{172}~\ding{181} (1~10)
	%无衬线加圈阳文数字:\ding{192}~\ding{201} (1~10)
	%xTODO:20160705 脚注不用上标
	%HACK:20160709 见 http://tex.stackexchange.com/questions/19844/how-to-set-superscript-footnote-mark-in-the-text-body-but-normalsized-in-the-foo
	\renewcommand{\thefootnote}{\ding{\numexpr191+\value{footnote} } }
	\makeatletter
	\newlength{\fnBreite}
	\renewcommand{\@makefntext}[1]{%
		\settowidth{\fnBreite}{\footnotesize\@thefnmark.i}
		\protect\footnotesize\upshape%
		\setlength{\@tempdima}{\columnwidth}\addtolength{\@tempdima}{-\fnBreite}%
		\makebox[\fnBreite][l]{\@thefnmark\phantom{  }}%
%		\parbox[t]{\@tempdima}{\everypar{\hspace*{1em}}\hspace*{-1em}\upshape#1}
		}
	\makeatother

%%颜色
\usepackage[svgnames]{xcolor}

%%图形
\usepackage{graphicx}

%%TeX Logos
\usepackage{hologo}

%%代码抄录
%\usepackage{shortvrb}
%	\MakeShortVerb |
\usepackage{fancyvrb}
	\VerbatimFootnotes

%%大段代码排版
\usepackage{listings}

%%彩色、加框环境
\usepackage[listings]{tcolorbox}

%%定制列表环境
\usepackage{enumitem}
	%定义带缩进的左对齐格式
	%前一个参数 2.1 em 确定标签的位置
	%后一个参数 1.2 em 确定标签与文字的距离
	%HACK:20160709 可能与字体、字号、标签内容有关
	\SetLabelAlign{leftalignwithindent}{\hspace{2.1 em} \makebox[1.2 em][l]{#1}}

%%定理类环境
%\usepackage[thmmarks, amsmath]{ntheorem}
%	\theoremstyle{plain} %带编号
%%	\theoremheaderfont{\myHeavy}
%	\theorembodyfont{\normalfont}
%%	\theoremsymbol{}
%%	\theoremsymbol{\ensuremath{\triangleleft}}	
%		\newtheorem{_myQuestion}{例}[chapter]
%		\newtheorem{_myAnswer}{答}[chapter]

%%索引
\usepackage{imakeidx}
	\makeindex[
		name = pkg,
		title = {宏包索引}
	]
	\makeindex[
		name = cmd,
		title = {命令、选项索引}
	]
	\newcommand{\BH}[1]{\hyperpage{#1}}
%	\let \oldindex = \index
%	\renewcommand{\index}[1]{\oldindex{#1|BH}}
	\newcommand{\indexpkg}[1]{\index[pkg]{#1@\pkg{#1}|BH} }
	\newcommand{\indexcmd}[1]{\index[cmd]{#1@\texttt{#1}|BH} }


\usepackage[numbers]{natbib}%\raggedleft
\bibliographystyle{mybst}

%TODO——环境
%%问题 Question & Answer (这里的空行用来分段)
%使用计数器 question,全文统一编号
\newcounter{question}
\setcounter{question}{1}
\renewcommand{\thequestion}{\arabic{question}}
\newenvironment{myQA}[1]
	{\addcontentsline{toc}{section}{\numberline {\thequestion}#1}
		{\large \sffamily \noindent \thequestion.~#1}
		
	}
	{
		
		\mbox{} \stepcounter{question}
	}

%%代码框
%可选参数表示 escape char, 默认为 `
%FIXME:20160705 不可以使用 tab 键,见 plain.tex 第 18 行
\newtcblisting{myCode}[1][`]
	{
		colback = AliceBlue, %需要 xcolor 宏包 svgnames 选项
%		colframe=yellow!50!black,
		sharp corners,
		boxrule = 0 cm,
		left = 0 cm, right = 0.5 cm, top = 0 cm, bottom = 0 cm,
		capture = hbox, %自动调整边框大小
		listing only,
		listing options = {
			language = tex,
			basicstyle = \ttfamily,
			keywordstyle = \normalsize,
			numbers = left,
			numberstyle = \itshape \footnotesize \color{gray},
			escapechar = #1
		}
	}

%\lstnewenvironment{myCode}[1][`]
%	{
%		\lstset{
%			language = [LaTeX] TeX,
%			basicstyle = \ttfamily,
%			keywordstyle = \normalsize,
%%			由于 80 字符限制,该选项可略去
%%			breaklines = true,
%			tabsize = 4,
%%			numbers = left,
%%			numberstyle = \itshape \footnotesize \color{gray},
%%			numbersep = 10 pt,
%%			该项无效
%%			showstringspaces = true,
%			escapechar = #1
%		}
%	}
%	{}

%%供索引使用的抄录环境
%\makeatletter
%\newenvironment{indexverb}{%
%	\def\verbatim@processline{\the\verbatim@line\ }%
%	\def\@verbatim{\the\every@verbatim
%		\obeylines
%		\let\do\@makeother \dospecials
%		\verbatim@font
%	}%
%	\def\endverbatim{\endgroup}%
%	\noindent
%	\verbatim}{\endverbatim}
%\makeatother

%%定制列表(不编号)
\newenvironment{myItemize}
	{\begin{enumerate} [
		label=\bullet, %标签样式:圆点
		align = leftalignwithindent, %对齐(见上)
		listparindent = 2 em, %条目段落缩进
		leftmargin = 0 pt, %文字左边距
		topsep = 0 pt,
		itemsep = 0 pt,
		parsep = 0 pt
	]}
	{\end{enumerate}}

%TODO——命令
%%宏包名
\newcommand{\pkg}[1]{{\SourceSans #1}}
%%文件名
\newcommand{\filename}[1]{{\SourceSans #1}}
%%参考文献
\newcommand{\myRef}[1]{\noindent
	{\sc Reference} \raisebox{-0.25 ex}{\ding{43}}
	\quad #1}
%%(La)TeX
%From hologo 宏包, 有修改
%FIXME:20160708 与汉字连用时之前不要加空格
\makeatletter
\DeclareRobustCommand{\LaTeXTeX}{
	(%
	\kern-.1em%
	L%
	\kern-.36em%
	{%
		\sbox\z@ T%
		\vbox to\ht0{%
			\hbox{%
				$\m@th$%
				\csname S@\f@size\endcsname
				\fontsize\sf@size\z@
				\math@fontsfalse
				\selectfont
				A%
			}%
			\vss
		}%
	}%
	\kern-.15em%
	)%
	\kern-.1em%
	\TeX
}
\makeatother
%%空行
\newcommand{\blankline}{\mbox{}}


%TODO——标题页
\title{
	\vspace{-4 cm} \color{Sienna} \Huge \LaTeX 排版笔记
}
\author{
	\CJKfamily{楷体} \color{DarkRed} \Large 曾祥东
}
\date{
	\CJKfamily{楷体} \color{Goldenrod} \Large \today
}

\begin{document}
	\frontmatter
	\maketitle
	
	\tableofcontents
	
	\mainmatter
	\chapter{文本}
		\begin{myQA}{如何优雅地在科技文献中使用句点?}
在一般中文文章中,通常使用句号“{\CJKfamily{宋体}。}”。在 Unicode 中,
它的代码为 \texttt{U+3002}。而在科技文献中,为了与一些下标区分,
经常使用句点“.”(即全角英文句号)
\footnote{在新的《标点符号用法:GB/T 15834—2011》
	\sscite{GB/T15834-2011标点符号} 
	中,这一条被删去了。是否采纳,您看着办。},
它的 Unicode 代码为 \texttt{U+FF0E}。
类似符号还有:半角英文句号达到 \texttt{U+002E}“.”,
半角中文句号 \texttt{U+FF61}
\footnote{本文所用字体并不包含该符号,可以脑补一下,
	或者改用微软雅黑等字体来显示。}%
等。这篇文章可以假装是科技文献,因此用了句点。

我们有三种使用句点的方案。

\emph{普青方案}:使用 \pkg{xeCJK} \indexpkg{xeCJK} 宏包的朋友,
在调用字体时使用
\code|Mapping = fullwidth-stop| \indexcmd{Mapping} 选项,
就可以将正常句号转换成句点。选项 \code|Mapping = full-stop|
的作用恰恰相反。

\emph{文青方案}:在导言区添加如下代码即可。此代码第一行将
“{\CJKfamily{宋体}。}”设置为活动符,并将其定义为命令,从而输出一个句点。
\begin{myCode}
\catcode`\*|{\CJKfamily{宋体}。}|* = \active
\newcommand{*|{\CJKfamily{宋体}。}|*}{.}
\end{myCode}
\indexcmd{\backslash catcode} \indexcmd{\backslash active}

\emph{二青方案}:利用编辑器全文替换。缺点是这种方式并不优雅。

\myRef{\citet{PKG2016xeCJK}}
\end{myQA}

%HACK:20160708 PDF 标签显示仍为句号,这里改用句点
\begin{myQA}{各种连字符、破折号的区别.}
在英文中主要有三种:hyphen、en dash 和 em dash。

\begin{myItemize}
\item  hyphen,-,即连字符,Unicode 代码为 \texttt{U+002D}
\footnote{严格来说,\texttt{U+002D} 表示“连字符或减号”。
	Unicode 还单独定义了一个连字符 \texttt{U+2010},
	但是输入不太方便(\code|\symbol{8208}|)。},
用普通键盘就可以直接输入。

\item en dash,--,即字母“n”宽度的破折号,
Unicode 代码为 \texttt{U+2013}。
在\LaTeXTeX 中,输入 \code|--| (即两个 hyphen)可以通过预先
设定的连字功能得到 en dash。

\item em dash ,---,想必就是字母“m”宽度的破折号,
Unicode 代码为 \texttt{U+2014}。
可以通过输入 \code|---| (即三个 hyphen)来得到。
\end{myItemize}

\blankline

中文中的连字符也有三种:短横线“{\Songti -}”、一字线“—”和浪纹线“~”。
短横线就是上文的 hyphen,而一字线则是 em dash
\footnote{观察仔细的话,可以发现它们其实略有差异。
	这是由于上文的 hyphen 和 em dash 使用了英文字体,
	而这里的短横线和一字线使用了中文字体。}。
剩下的浪纹线,Unicode 代码为 \texttt{U+FF5E}。它的输入比较麻烦,
因为\LaTeXTeX 不是原生支持 Unicode 的。较安全的方法是用
\code|\symbol{65374}| \indexcmd{\backslash symbol}输入。
如果使用 \hologo{XeTeX} 编译,直接在源代码中输入该符号也可以得到。

中文中的破折号是“——”,实际上就是两个连在一起的 em dash。
使用中文输入法,一般很容易输入。顺便一说,之前的一字线,
可以由破折号删掉一半得到。这实际上是最简单的方法。
%TODO:20160828 适用范围、参考文献
\end{myQA}

\begin{myQA}{如何在正文中使用上下标?}
上下标的命令分别是 \code|\textsuperscript|
和 \code|\textsubscript|
\footnote{在早期版本的 \LaTeX 中,
	使用 \code|\textsubscript| 命令可能需要调用
	\pkg{fixltx2e} \indexpkg{fixltx2e} 宏包。}。
\indexcmd{\backslash textsuperscript}
\indexcmd{\backslash textsubscript}
注意要加分组括号,否则这两个命令只会对其后的第一个字符起作用。

利用行内公式,也可以实现上下标(请注意空格):
\begin{myExampleV}
{\vspace{1 ex}

这是第一个 $^\text{上标}$ 啦啦啦。
This is another $^\text{supersciprt}$ lalala.

这是第三个$^\text{上标}$啦啦啦。
This is the fourth$^\text{supersciprt}$lalala.

这是一个$_\text{下标}$。
This is a $_\text{subscript}$.}
%% 需要使用 `amsmath' 宏包
这是第一个*|\vispace|*$^\text{上标}$*|\vispace|*啦啦啦。
This is another*|\vispace|*$^\text{supersciprt}$*|\vispace|*lalala.

这是第三个$^\text{上标}$啦啦啦。
This is the fourth$^\text{supersciprt}$lalala.

这是一个$_\text{下标}$。
This is a*|\vispace|*$_\text{subscript}$.
\end{myExampleV}
在中文环境中,上下标前后的空格会自动加上,
无论是不是手动添加。有的时候这会造成麻烦。
而且这种手段略显猥琐,不推荐使用。

\myRef{\citet{TSE2012superscript,TSE2010subscript}}
\end{myQA}

\begin{myQA}{如何输入“naïve”这样的单词?}
在 \hologo{XeTeX} 这样的排版引擎下,只需将文件编码设置为 UTF-8,
就可以直接输入:
\begin{myExampleV}
{\vspace{1 ex}

naïve \quad café \quad Schrödinger \quad Ürümqi(Wūlǔmùqí)}
naïve \quad café \quad Schrödinger \quad Ürümqi(Wūlǔmùqí)
\end{myExampleV}
除了拉丁语拓展,实际上在大多数字母都可以用这种方法直接输入
(前提是有合适的字体支持):
\begin{myExampleV}
{\vspace{1 ex}

{\LinLibertine Пу́тин \quad	 Σωκράτης \quad אברהם}}
%% 需要使用 `fontspec' 宏包
\setmainfont{Linux Libertine O}
*|{\CourierNew Пу́тин}|*     \quad  % 普京(俄语)
*|{\CourierNew Σωκράτης}|*  \quad  % 苏格拉底(希腊语)
*|{\CourierNew אברהם}|*            % 亚伯拉罕(希伯来语)
\end{myExampleV}
\end{myQA}
		
	\chapter{数学与公式}
		\begin{myQA}{如何将实部、虚部用 Re 和 Im 表示?}
	在\LaTeXTeX 中,命令 \verb|\Re| 和 \verb|\Im| 得到的是大写哥特体字母
	$\Re$ 和 $\Im$。这是高德纳在 \filename{plain.tex} 中定义的。
	
	用下面的命令可以将它们重定义为更常用的格式 $\operatorname{Re}$
	和 $\operatorname{Im}$:
\begin{myCode}[|]
\renewcommand{\Re}{\operatorname{Re}}
\renewcommand{\Im}{\operatorname{Im}}
\end{myCode}
	
	使用 \pkg{physics} 中的 \verb|\Re| 和 \verb|\Im| 也可以达到同样的目的。
	这个宏包还把原来的哥特体保存在了命令 \verb|\real| 和 \verb|\imaginary| 中。
	
	\myRef{115,116}
\end{myQA}
		
	\chapter{字体}
	\chapter{图形}
	
	\backmatter
	\bibliography{Reference}
	\printindex[pkg]
	\printindex[cmd]
\end{document}
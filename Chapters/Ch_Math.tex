\begin{myQA}{如何将实部、虚部用 Re 和 Im 表示?}
在\LaTeXTeX 中,命令 \code|\Re| 和 \code|\Im| 得到的是大写哥特体字母
$\Re$ \indexcmd{\backslash Re} 和 $\Im$ \indexcmd{\backslash Im}。
这是高德纳在 \filename{plain.tex} 中定义的。

用下面的命令可以将它们重定义为更常用的格式 $\operatorname{Re}$
和 $\operatorname{Im}$:
\begin{myCode}
\renewcommand{\Re}{\operatorname{Re}}
\renewcommand{\Im}{\operatorname{Im}}
\end{myCode}

使用 \pkg{physics} \indexpkg{physics} 宏包中的 \code|\Re| 和
\code|\Im| 也可以达到同样的目的。
这个宏包还把原来的哥特体保存在了命令 \code|\real| 和 \code|\imaginary| 中。
\indexcmd{\backslash real} \indexcmd{\backslash imaginary}

\myRef{\citet{PKG2012physics}}
\end{myQA}

\begin{myQA}{怎样正确使用微分符号?}
按照传统,$\dd{x}$、$\dd{\theta}$ 等微元的前后均需要留出一个细空格
(thin space)的距离。在 \LaTeX 中,细空格用命令 \code|\,| 表示,
默认情况下相当于 \SI{3}{mu} 的长度。这个 \si{mu} 表示数学单位,
一个 \si{mu} 等于 $1/18$ 个 \si{em} 的长度。至于 \si{em},假设你是知道的。

下面是一个例子:
\begin{myExampleV}
{\begin{alignat*}{2}
	&\text{错误的:} \quad \iint f(x,y) \mathrm{d}x \mathrm{d}y
	&\quad\quad& \iiint \mathrm{d}r \mathrm{d}\varphi
		\mathrm{d}\theta \\
	&\text{正确的:} \quad \iint f(x,\,y) \, \mathrm{d}x \, \mathrm{d}y
	&& \iiint \mathrm{d}r \, \mathrm{d}\varphi \, \mathrm{d}\theta
\end{alignat*}}
\begin{alignat*}{2}
    &\text{错误的:} \quad
        \iint f(x,y) \mathrm{d}x \mathrm{d}y &\quad \quad&
        \iiint \mathrm{d}r \mathrm{d}\varphi \mathrm{d}\theta \\
    &\text{正确的:} \quad
        \iint f(x,\,y) \, \mathrm{d}x \, \mathrm{d}y &&
        \iiint \mathrm{d}r \, \mathrm{d}\varphi \, \mathrm{d}\theta
\end{alignat*}
\end{myExampleV}
但是“\code|\,|”一多,写起来很累,而且容易错。惯例,可以新搞一个定义:
\begin{myExampleH}[0.25]
{\newcommand{\dif}{\operatorname{d \!}}
\begin{gather*}
	\iint f(x, \, y) \dif x \dif y \\
	\frac{\dif f}{\dif x} = \sin x \\
	\frac{\dif{^2 g}}{\dif{x^2}} = \cos x
\end{gather*}}
%% ----------导言区---------- %%
\DeclareMathOperator{\dif}{d \!}
% 或者 \newcommand{\dif}{\operatorname{d \!}}
%% ----------导言区---------- %%

\begin{gather*}
    \iint f(x, \, y) \dif x \dif y \\
    \frac{\dif f}{\dif x} = \sin x \\
    \frac{\dif{^2 g}}{\dif{x^2}} = \cos x
\end{gather*}
\end{myExampleH}
这里给出了两种实现方法。如果用 \code|\DeclareMathOperator| 的话,
要注意该声明的代码只能放在导言区。另外这个命令需要
\pkg{amsmath} \indexpkg{amsmath} 宏包的支持。

这两种方法的原理是类似的。利用数学算子
(\code|\DeclareMathOperator| 或 \code|\operatorname|)
命令保证了微分算子前面的间距,又用“\code|\!|”去掉了它后面的间距。
同时,这个命令还自动把“$\mathrm{d}$”切换到了直立字体(\code|\mathrm|)。

如果微分算子“$\mathrm{d}$”后面跟的不是字母,那就不要偷懒,
该加的分组括号不能少,要不然 bug 会来得出人意料:
\begin{myExampleV}
{\newcommand{\dif}{\operatorname{d \!}}
\begin{alignat*}{4}
	&\text{不好的:} &\quad&
		\dif (\cos x) &\quad \quad&
		\dif \left(\cos x\right) &\quad \quad&
		\dif \left(\frac{\ln x}{x}\right) \\
	&\text{好的:} &&
		\dif{(\cos x)} &&
		\dif{\left(\cos x\right)} &&
		\dif{\left(\frac{\ln x}{x}\right)}
\end{alignat*}}
\newcommand{\dif}{\operatorname{d \!}}

\begin{alignat*}{4}
    &\text{不好的:} &\quad&
        \dif (\cos x) &\quad \quad&
        \dif \left(\cos x\right) &\quad \quad&
        \dif \left(\frac{\ln x}{x}\right) \\
    &\text{好的:} &&
        \dif{(\cos x)} &&
        \dif{\left(\cos x\right)} &&
        \dif{\left(\frac{\ln x}{x}\right)}
\end{alignat*}
\end{myExampleV}

普通括号跟在微分算子“$\mathrm{d}$”后面,间距太小,不好看。
但是如果是定界符括号,间距却又是正常的。所以还是老老实实加上“\code|{}|”。

\blankline

根据 ISO 80000-2 的要求,微分算子应使用直立的“$\mathrm{d}$”。
但是,如果你就是任性,非要用斜体的“$d$”,也是可以的。
把之前代码中的 \code|d| 用 \code|\mathnormal{d}| 来代替,
就可以强制使用斜体(当然前提要求默认数学字体就是倾斜的)。

\blankline

\pkg{physics} \indexpkg{physics} 宏包中定义了 \code|\differential| 命令
(简写为 \code|\dd|),它涵盖了之前我们做的事情,又通过可选参数引入了上标。
对于圆括号,它还给出了自动处理的解决方案:
\begin{myExampleH}
{\begin{equation*}
	\dd x \quad \dd{\theta} \quad \dd[2]{x} \quad
	\dd(\sin \theta) \quad \dd[2](\frac{\ln x}{x})
\end{equation*}}
%% ----------导言区---------- %%
\usepackage{physics}
%% ----------导言区---------- %%

\begin{equation*}
    \dd x \quad
    \dd{\theta} \quad
    \dd[2]{x} \quad
    \dd(\sin \theta) \quad
    \dd[2](\frac{\ln x}{x})
\end{equation*}
\end{myExampleH}

另外,这个宏包提供的 \code|\derivative| 命令(简写为 \code|\dv|)
可以类似的手法处理导数:
\begin{myExampleH}
{\begin{equation*}
	\dv{x} \quad \dv{R}{\theta} \quad \dv[n]{f}{x} \quad
	\dv{r}(\frac{\ln r}{r})
\end{equation*}}
%% ----------导言区---------- %%
\usepackage{physics}
%% ----------导言区---------- %%

\begin{equation*}
    \dv{x} \quad
    \dv{R}{\theta} \quad
    \dv[n]{f}{x} \quad
    \dv{r}(\frac{\ln r}{r})
\end{equation*}
\end{myExampleH}

\myRef{\citet{PKG2012physics}}
\end{myQA}

\begin{myQA}{如何优雅地在科技文献中使用句点?}
在一般中文文章中,通常使用句号“{\CJKfamily{宋体}。}”。在 Unicode 中,
它的代码为 \texttt{U+3002}。而在科技文献中,为了与一些下标区分,
经常使用句点“.”(即全角英文句号)
\footnote{在新的《标点符号用法:GB/T 15834—2011》
	\sscite{GB/T15834-2011标点符号} 
	中,这一条被删去了。是否采纳,您看着办。},
它的 Unicode 代码为 \texttt{U+FF0E}。
类似符号还有:半角英文句号达到 \texttt{U+002E}“.”,
半角中文句号 \texttt{U+FF61}
\footnote{本文所用字体并不包含该符号,可以脑补一下,
	或者改用微软雅黑等字体来显示。}%
等。这篇文章可以假装是科技文献,因此用了句点。

我们有三种使用句点的方案。

\emph{普青方案}:使用 \pkg{xeCJK} \indexpkg{xeCJK} 宏包的朋友,
在调用字体时使用
\code|Mapping = fullwidth-stop| \indexcmd{Mapping} 选项,
就可以将正常句号转换成句点。选项 \code|Mapping = full-stop|
的作用恰恰相反。

\emph{文青方案}:在导言区添加如下代码即可。此代码第一行将
“{\CJKfamily{宋体}。}”设置为活动符,并将其定义为命令,从而输出一个句点。
\begin{myCode}
\catcode`\*|{\CJKfamily{宋体}。}|* = \active
\newcommand{*|{\CJKfamily{宋体}。}|*}{.}
\end{myCode}
\indexcmd{\backslash catcode} \indexcmd{\backslash active}

\emph{二青方案}:利用编辑器全文替换。缺点是这种方式并不优雅。

\myRef{\citet{PKG2016xeCJK}}
\end{myQA}

%HACK:20160708 PDF 标签显示仍为句号,这里改用句点
\begin{myQA}{各种连字符、破折号的区别.}
在英文中主要有三种:hyphen、en dash 和 em dash。

\begin{myItemize}
\item  hyphen,-,即连字符,Unicode 代码为 \texttt{U+002D}
\footnote{严格来说,\texttt{U+002D} 表示“连字符或减号”。
	Unicode 还单独定义了一个连字符 \texttt{U+2010},
	但是输入不太方便(\code|\symbol{8208}|)。},
用普通键盘就可以直接输入。

\item en dash,--,即字母“n”宽度的破折号,
Unicode 代码为 \texttt{U+2013}。
在\LaTeXTeX 中,输入 \code|--| (即两个 hyphen)可以通过预先
设定的连字功能得到 en dash。

\item em dash ,---,想必就是字母“m”宽度的破折号,
Unicode 代码为 \texttt{U+2014}。
可以通过输入 \code|---| (即三个 hyphen)来得到。
\end{myItemize}

\blankline

中文中的连字符也有三种:短横线“{\Songti -}”、一字线“—”和浪纹线“~”。
短横线就是上文的 hyphen,而一字线则是 em dash
\footnote{观察仔细的话,可以发现它们其实略有差异。
	这是由于上文的 hyphen 和 em dash 使用了英文字体,
	而这里的短横线和一字线使用了中文字体。}。
剩下的浪纹线,Unicode 代码为 \texttt{U+FF5E}。它的输入比较麻烦,
因为\LaTeXTeX 不是原生支持 Unicode 的。较安全的方法是用
\code|\symbol{65374}| \indexcmd{\backslash symbol}输入。
如果使用 \hologo{XeTeX} 编译,直接在源代码中输入该符号也可以得到。

中文中的破折号是“——”,实际上就是两个连在一起的 em dash。
使用中文输入法,一般很容易输入。顺便一说,之前的一字线,
可以由破折号删掉一半得到。这实际上是最简单的方法。
%TODO:20160828 适用范围、参考文献
\end{myQA}

\begin{myQA}{如何在正文中使用上下标?}
上下标的命令分别是 \code|\textsuperscript|
和 \code|\textsubscript|
\footnote{在早期版本的 \LaTeX 中,
	使用 \code|\textsubscript| 命令可能需要调用
	\pkg{fixltx2e} \indexpkg{fixltx2e} 宏包。}。
\indexcmd{\backslash textsuperscript}
\indexcmd{\backslash textsubscript}
注意要加分组括号,否则这两个命令只会对其后的第一个字符起作用。

利用行内公式,也可以实现上下标(请注意空格):
\begin{myExampleV}
{\vspace{1 ex}

这是第一个 $^\text{上标}$ 啦啦啦。
This is another $^\text{supersciprt}$ lalala.

这是第三个$^\text{上标}$啦啦啦。
This is the fourth$^\text{supersciprt}$lalala.

这是一个$_\text{下标}$。
This is a $_\text{subscript}$.}
%% 需要使用 `amsmath' 宏包
这是第一个*|\vispace|*$^\text{上标}$*|\vispace|*啦啦啦。
This is another*|\vispace|*$^\text{supersciprt}$*|\vispace|*lalala.

这是第三个$^\text{上标}$啦啦啦。
This is the fourth$^\text{supersciprt}$lalala.

这是一个$_\text{下标}$。
This is a*|\vispace|*$_\text{subscript}$.
\end{myExampleV}
在中文环境中,上下标前后的空格会自动加上,
无论是不是手动添加。有的时候这会造成麻烦。
而且这种手段略显猥琐,不推荐使用。

\myRef{\citet{TSE2012superscript,TSE2010subscript}}
\end{myQA}

\begin{myQA}{如何输入“naïve”这样的单词?}
在 \hologo{XeTeX} 这样的排版引擎下,只需将文件编码设置为 UTF-8,
就可以直接输入:
\begin{myExampleV}
{\vspace{1 ex}

naïve \quad café \quad Schrödinger \quad Ürümqi(Wūlǔmùqí)}
naïve \quad café \quad Schrödinger \quad Ürümqi(Wūlǔmùqí)
\end{myExampleV}
除了拉丁语拓展,实际上在大多数字母都可以用这种方法直接输入
(前提是有合适的字体支持):
\begin{myExampleV}
{\vspace{1 ex}

{\LinLibertine Пу́тин \quad	 Σωκράτης \quad אברהם}}
%% 需要使用 `fontspec' 宏包
\setmainfont{Linux Libertine O}
*|{\CourierNew Пу́тин}|*     \quad  % 普京(俄语)
*|{\CourierNew Σωκράτης}|*  \quad  % 苏格拉底(希腊语)
*|{\CourierNew אברהם}|*            % 亚伯拉罕(希伯来语)
\end{myExampleV}
\end{myQA}